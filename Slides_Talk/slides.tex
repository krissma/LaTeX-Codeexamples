%% Begin slides template file
\documentclass[11pt,t,usepdftitle=false,aspectratio=169]{beamer}
%% ------------------------------------------------------------------
%% - aspectratio=43: Set paper aspect ratio to 4:3.
%% - aspectratio=169: Set paper aspect ratio to 16:9.
%% ------------------------------------------------------------------

\usetheme[nototalframenumber,foot,logo]{uibk}
\usepackage{todonotes}
\usepackage{hyperref}
\presetkeys{todonotes}{inline}{}


\newcommand{\Kristina}[1]{\textcolor{violet}{\textbf{\textit{<#1>}}}}
%% ------------------------------------------------------------------
%% - foot: Add a footer line for conference name and date.
%% - logo: Add the university logo in the footer (only if 'foot' set).
%% - bigfoot/sasquatch: Larger font size in footer.
%% - nototalslidenumber: Hide the total number of slides (only if 'foot' set)
%% - license: Add CC-BY license symbol to title slide (e.g., for conference uploads)
%%   (TODO: At the moment no other licenses are supported.)
%% - licenseall: Add CC-BY license symbol to all subsequent slides slides
%% - url: use \url{} rather than \href{} on the title page
%% - nosectiontitlepage: switches off the behaviour of inserting the
%%   titlepage every time a \section is called. This makes it possible to
%%   use more than one section + thanks page and a ToC off by default.
%%   If the 'nosectiontitlepage' is set you can create UIBK title slides
%%   using the command '\uibktitlepage{}' in your document to create
%%   one or multiple title slides.
%% ------------------------------------------------------------------

%% ------------------------------------------------------------------
%% The official corporate colors of the university are predefined and
%% can be used for e.g., highlighting something. Simply use
%% \color{uibkorange} or \begin{color}{uibkorange} ... \end{color}
%% Defined colors are:
%% - uibkblue, uibkbluel, uibkorange, uibkorangel, uibkgray, uibkgraym, uibkgrayl
%% The frametitle color can be easily adjusted e.g., to black with
%% \setbeamercolor{titlelike}{fg=black}
%% ------------------------------------------------------------------

%\setbeamercolor{verbcolor}{fg=uibkorange}
%% ------------------------------------------------------------------
%% Setting a highlight color for verbatim output such as from
%% the commands \pkg, \email, \file, \dataset 
%% ------------------------------------------------------------------


%% information for the title page ('short title' is the pdf-title that is shown in viewer's titlebar)
%% information for the title page ('short title' is the pdf-title that is shown in viewer's titlebar)
\title{Learn to Love {\LaTeX}}

\author{Kristina Magnussen}
%('short author' is the pdf-metadata Author)
%% If multiple authors are required and the font size is too large you
%% can overrule the font size of author and url by calling:
%\setbeamerfont{author}{size*={10pt}{10pt},series=\mdseries}
%\setbeamerfont{url}{size*={10pt}{10pt},series=\mdseries}
%\URL{}
%\subtitle{}

\footertext{{\LaTeX} beamer theme}
\date{2017-07-25}

\headerimage{3}
%% ------------------------------------------------------------------
%% The theme offers four different header images based on the
%% corporate design of the university of innsbruck. Currently
%% 1, 2, 3 and 4 is allowed as input to \headerimage{...}. Default
%% or fallback is '1'.
%% ------------------------------------------------------------------

\begin{document}
	
	%% ALTERNATIVE TITLEPAGE
	%% The next block is how you add a titlepage with the 'nosectiontitlepage' option, which switches off
	%% the default behavior of creating a titlepage every time a \section{} is defined.
	%% Then you can use \section{} as it's originally intended, including a table of contents.
	% \usebackgroundtemplate{\includegraphics[width=\paperwidth,height=\paperheight]{titlebackground.pdf}}
	% \begin{frame}[plain]
		%     \titlepage
		% \end{frame}
	% \addtocounter{framenumber}{-1}
	% \usebackgroundtemplate{}}

%% Table of Contents, if wanted:
%% this requires the 'nosectiontitlepage' option and setting \section{}'s as you want them to appear here.
%% Subsections and subordinates are suppressed in the .sty at the moment, search
%% for \setbeamertemplate{subsection} and replace the empty {} with whatever you want.
%% Although it's probably too much for a presentation, maybe for a lecture.
% \begin{frame}
	%     \vspace*{1cm plus 1fil}
	%     \tableofcontents
	%     \vspace*{0cm plus 1fil}
	% \end{frame}


%% this sets the first PDF bookmark and triggers generation of the title page
\section{Bookmark Title}

%% this just generates PDF bookmarks

%% first slide

\begin{frame}{What is LaTeX?}
	\begin{itemize}
		\item LaTeX is a system for typesetting
		\item The goal is for authors to focus less on design and more on content
		\item Very well suited for large documents and scientific work
		\item Helpful when representing mathematical formulas
		\item Can be used for slides, papers, posters etc. 
	\end{itemize}
\end{frame}



\begin{frame}{What I've Learned}
	\begin{itemize}
		\item Reuse reuse reuse $\rightarrow$ Recycle your own code and all the code you can get from other people 
		\item Error messages are often useless \\
		$\rightarrow$ Compile often and try to retrace your steps 
		\item Small commands, e.g. write a \textasciitilde{} before \texttt{\textbackslash{}ref} and \texttt{\textbackslash{}cite} to avoid line breaks 
		\item When including images, use pdf instead of png, jpeg etc. for better quality 
	\end{itemize}
	
\end{frame}


\begin{frame}{The todonotes packages}
	Include: \texttt{\textbackslash{}usepackage\{todonotes\}} \\
	When using the beamer package, you also need this line: \\
	\texttt{\textbackslash{}presetkeys\{todonotes\}\{inline\}\{\}}
	\todo{TODO: Finish this slide}
	\missingfigure{Insert a nice picture here}
	
\end{frame}



\begin{frame}{Custom Commands}
	LaTeX allows you to define custom commands, this can be useful if there is e.g. a sequence of text you will be repeating several times. \\ \medskip 
	\textbf{Example:} \\ \texttt{\textbackslash{}newcommand\{\textbackslash{}Kristina\}[1]\{\textbackslash{}textcolor\{violet\}\{\textbackslash{}textbf\{\textbackslash{}textit\textbackslash{}\{<\#1>\}\}\}\}} \\  \medskip
	
	\textbf{Usage:} \\
	\texttt{\textbackslash{}Kristina}\{My custom comment\} \\ \medskip
	
	\textbf{Result:} \\ 
	\Kristina{My custom comment}
\end{frame}


\begin{frame}{Tools}
	\begin{itemize}
		\item Editors, e.g. Overleaf (useful for collaboration), TeXstudio etc. 
		\item Mathpix: converts handwritten formulas to LaTeX code \url{https://mathpix.com/image-to-latex}
	\end{itemize}

\end{frame}

\begin{frame}{Resources}
	\begin{itemize}
		\item \href{https://blog.martisak.se/2020/05/03/top-ten-latex-packages/}{Interesting LaTeX packages}
     \item \href{https://www.hmc.edu/mathematics/wp-content/uploads/sites/49/2019/06/latex-hints.pdf}{Latex hints} \\ 
	\item \href{https://guides.lib.chalmers.se/overleaf_latex/tools}{Cheatsheets and tools for Overleaf}
	\item \href{https://typeset.io/resources/the-only-latex-editor-guide-you-will-need/}{Latex editor guide} \\
	\item \href{https://waterprogramming.wordpress.com/2021/10/05/make-latex-easier-with-custom-commands/}{How to define custom commands}
	\item \href{https://git.uibk.ac.at/uibklatex/beamer\_letter}{UIBK LaTeX templates} \\ \textbf{$\rightarrow$ Download from the git and not the university website!}
\end{itemize}
\end{frame}




%% to show a last slide similar to the title slide: information for the last page
\title{Thank you for your attention!}
\subtitle{}
\section{Thanks}

\end{document}

